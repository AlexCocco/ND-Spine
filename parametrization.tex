The model target is represented by comfort-related simulations of the
vibrational response of the upper body of rotorcraft pilots and passengers. In
this context, the model should be able to reproduce the \emph{average}, or
\emph{most plausible} subject possessing a certain set of biometric
characteristics, whilst it it is not useful to build a purely subject-specific
model.   

The model is fully parametrized and can be generated starting from a generic
anthropometric dataset $\mathbf{s}$ comprising age, $a$, Body Mass Index
$\mathrm{BMI}$, stature $h$ and gender $g$:
\begin{equation}
	\mathbf{s} = \begin{bmatrix} a &\mathrm{BMI} &h &g \end{bmatrix}^{T}
	\label{eq:anthro_vect}
\end{equation}
A statistical model published by Shi et al.~\cite{SHI-2014-JB} is used to
generate the most plausible ribcage geometry associated with the anthropometric
parameters. The model of Shi et al. was derived from the analysis of thorax CT
scans of 89 subjects, divided in 8 age groups and of both sexes. Image
segmentation allowed to identify the position of 464 landmarks on the left side
of each subject's ribcage (Cfr. Fig.~\ref{fig:ribcage-3d}). The positions of the
landmarks were subjected to a PCA (Principal Component Analysis) following the
procedure outlined by Allen et al.\ in \cite{ALLEN-2003-ACMTC}. 

\begin{figure}[htbp]
	\centering
	\includegraphics[width=.8\textwidth]{\gfxdir ribcage}
	\caption{Example of the generated ribcage landmarks point cloud, using
	the model published by Shi et al.~\cite{SHI-2014-JB}. Values on the axes
	are in mm.}
	\label{fig:ribcage-3d}
\end{figure}

After generating the ribcage landmarks, a bounding box is fitted to them and its
dimensions $x, y, z$ are compared to those of the reference subject used in the
work of Kitazaki and Griffin~\cite{KITAZAKI1997} to yield three scaling
coefficients:
\begin{subequations}
	\begin{align}
		\lambda_x &= \dfrac{x}{x_0} &\lambda_y &= \dfrac{y}{y_0}
		&\lambda_z &= \dfrac{z}{z_0}
	\end{align}
	\label{eq:scalingf}
\end{subequations}
Since~\cite{KITAZAKI1997} does not report the anthropometric characteristics of
the subject, they were estimated a-posteriori by searching through an Sequential
Quadratic Programming optimisation algorithm the set $\mathbf{s}_0$ that
minimises the mean square error of the distances between the estimated
landmarks, obtained through the ribcage model, closer to the vertebr\ae
transverse processes and their corresponding locations estimated from the data
of Kitazaki and Griffin. The resulting reference subject is a 34 years old male,
1.78 m tall and weighting 84 kilograms, for a BMI of approximately 26.5.

Using as reference the \emph{normal} configuration found
in~\cite{KITAZAKI1997}, the initial positions of nodes for the modelled subject
are obtained by directly scaling their components with the coefficients
$\lambda_i$.

Stiffnesses are instead scaled comparing them to \emph{equivalent} lumped axial
and bending stiffnesses as follows: 
\begin{subequations}
	\begin{align}
		K_a' \sim \dfrac{EA'}{L'} 
			& = \dfrac{EA\lambda_x\lambda_y}{L\lambda_z} 
	\end{align}
\end{subequations}

\subsection{Validation with experimental result}

\begin{table}[htbp]
  \centering
  
    \begin{tabular}{lllll}
    \textbf{Model} & \textbf{Age } & \textbf{Weight [Kg]} & \textbf{Stature [m]} & \textbf{BMI [kg/m\^2]} \\
    \hline
    \textbf{Min} & 33,7  & 54,88 & 1,591 & 21,681 \\
    \textbf{Mean} & 33,7  & 69,5  & 1,71  & 23,768 \\
    \textbf{Max} & 33,7  & 83,9  & 1,823 & 25,246 \\
    \end{tabular}%
    \caption{Antropometric parameters used for the comparison}
  \label{tab:addlabel}%
\end{table}%



\begin{figure}[htbp]
	\centering
	\includegraphics[width=\textwidth]{\gfxdir scaling-no-backrest}
	\caption{Comparison between the experimental apparent mass from \cite{TOWARD2011} and the FEM one}
	\label{fig:scaling-no-backrest}
\end{figure}

\begin{figure}[htbp]
	\centering
	\includegraphics[width=\textwidth]{\gfxdir frf_3_axis}
	\caption{Seat to head transmissibility in the three directions: comparison between experimental result of \cite{Mandapuram2012}
	\label{fig:frf_3}}
\end{figure}


