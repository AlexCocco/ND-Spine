The spine model has been developed aiming at a high level of generality in its
applications.  As an encompassing example, in the present work the solution
phases needed to extract a reduced order model (ROM) for the evaluation of the
upper body vertical vibration response of seated rotorcraft pilot/passengers
will be outlined. The general procedure consists in several simulation steps: 

\begin{enumerate}
  \item an underdetermined inverse kinematics analysis that determines the pose
    of the spine in relation with the imposed position of the head and of the
    buttocks;
  \vspace{-0.2em}
  \item an inverse dynamics analysis that estimates the passive muscular
    intervertebral moments;
    \vspace{-0.2em}
  \item a direct dynamics analysis aimed at estimating the effect of the active
    muscular intervertebral moments;
    \vspace{-0.2em}
  \item an eigenanalysis, directly performed on the system of Differential-Algebraic Equations (DAE) system, to extract the ROM.
\end{enumerate}

To obtain a square problem in the kinematic inversion when solving for the 
system's positions, a series of static problems are set up, in which \emph{dummy} springs
act on the redundant degrees of freedom \cite{IK:FumagalliASME}. The
stiffnesses of the springs act as penalty coefficients for the motion of the degrees of 
freedom they are connected to. They can, for example, be crafted to minimize the norm 
of the internal bending moment in the sagittal plane, due to weight.

From the inverse kinematics analysis, the configuration of maximum ergonomy 
of the spine is obtained. In the subsequent direct analysis, active muscular
moments are estimated introducing simple controllers that
introduce intervertebral axial forces and bending moments linearly proportional
to the difference between the current vertebr\ae\ relative positions and the 
maximum ergonomy configuration.

Once the equilibrium position has been reached, an eigenanalysis is performed 
to extract a ROM \cite{RIPEPI-2011} of the spine suitable for vibration analysis
in the vertical direction, to be used in linearized, 
comprehensive rotorcraft vibration analysis \cite{TAMER-2017-ERF}.
Alternatively, the complete multibody model can be used in direct multibody 
bioaeroservoelastic analysis of the system \cite{MASARATI-2015-ERF}.
