The vibrational response of the human body is directly related to unpleasant sensations
(discomfort), degradation of efficiency in performing a task and may ultimately lead to
health related issues~\cite{griffin-1996-hbhv,garg-1976-ie3tsmc,hill-2009-jb}.
Thus, proper modeling of the human body response is of critical importance when evaluating
both vehicle comfort performance~\cite{deoliveira-2005-asem} and potential adverse
involuntary feedthrough of command inputs~\cite{quaranta-2013-jsv}, especially when,
as is the case in rotorcraft, they exhibit an intrinsic propensity to develop a high level
of vibrations.

Parameters of the vibratory response of the human body depend on its mechanical properties
and geometry, which in turn may depend on anthropometric
variables like age, gender, weight and stature. Therefore, the variance of the parameters
influencing the response has to be taken into account in the design stage, in order to
ensure an adequate level of robustness. Multibody modeling is a viable tool in this
context, since it can be viewed as first-principles approach. Starting from the
anthropometric parameters, through the geometrical and structural modeling, it supports the
extraction of the relevant synthetic parameters evaluating the fitness of a particular
design choice to the goal of achieving a greater pilot (or passenger) comfort
and/or of reducing the insurgence of possible triggers of adverse interaction
phenomena. 

This work is specifically focused on the multibody modeling of the human spine. A complete
model able to comprehend the spine behaviour has been developed in the
free general purpose multibody solver MBDyn, aimed at direct and inverse dynamics analyses,
and reduced order model extraction. Although at the current stage the model is
primarily used to estimate properties related to spine vertical dynamics in the
sagittal plane, it easy to reconfigure it to capture the full 3-dimensional
behavior of the spine.

As stated in the previous paragraphs, the model is aimed especially to
comfort-related modeling of rotorcraft. In this context, one of the most
important measures is the total force transmissibility between the human subject
(i.e. the pilot or the passenger of the analyzed rotorcraft) and the rotorcraft
itself. Therefore, a comparison between experimental data available in the
literature and the model results in terms of the mechanical impedance at the
interface point between the body and the seat has been carried out. \\
\textcolor{red}{TODO: completare}
