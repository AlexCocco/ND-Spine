% Figure
\begin{figure}[htbp]
  \centering
  \hfil
  \includegraphics[width=.171\textwidth]{\gfxdir spine_arms.png}
  \hfil
  \def\svgwidth{.3\textwidth}
  \input{\gfxdir vert_constr.eps_tex}
  \hfil
  \def\svgwidth{.3\textwidth}
  \input{\gfxdir vert_visc_constr.eps_tex} 
  \hfil
  \caption{Algebraic and deformable constraints connecting vertebr\ae\ nodes, indicated with
    	$V$, and viscer\ae\ nodes, indicated with $S$.}
  \label{fig:example}
\end{figure}

The model comprises 34 rigid bodies associated with nodes (i.e.\ entities
possessing degrees of freedom) placed in correspondence of vertebr\ae\ from S1 to C1, the
head, and of 8 visceral masses elastically connected to vertebr\ae\ from S1 to T10.
It follows the concepts proposed by Kitazaki and Griffin~\cite{KITAZAKI1997},
but extends the modeling to a full 3D representation of the upper body dynamics,
exploiting the database provided by Privitzer and
Belytschko~\cite{BELYTSCHKO-1978-AF-33615-76-C-0506}, whose sagittal plane data
was also used by Kitazaki and Griffin. 

Each vertebral section node is located in the center of the corresponding
vertebra's body, with axis $z$ aligned with the local tangent to the curve
described by the spine longitudinal axis, axis $y$ directed laterally and axis
$x$ pointing anteriorly (Cf. Fig.~\ref{fig:vertebra_RF}).

\begin{figure}[htbp]
	\centering
	\includegraphics[width=.3\textwidth]{\gfxdir vertebra_RF.png}
	\caption{Each node related to a vertebral section is located in the
	center of the corresponding vertebra's body, with axis $z$ along the
	local tangent to the spine axis curve and axis $y$ poiting laterally.
	\textcolor{red}{Figura non chiara: da rifare}}
	\label{fig:vertebra_RF}
\end{figure}

The spine itself is composed of 25 vertebral nodes connected by 24 linear
viscoelastic 3D elements, both acting on relative displacements and on relative
rotations between adjacent nodes. The head and the Sacrum are considered rigid
bodies and are also connected, respectively, to C1 and to L5 through a linear
viscoelastic elements. 

When restricted to the sagittal plane, vertebral nodes are connected to each other by
algebraic constraints limiting their relative degrees of freedom to sliding along the
spine axis and rotation about the lateral axis. When the model is used to perform
3-dimensional simulation, all of the vertebr\ae\ nodes relative rotation degrees of freedom
are unconstrained, while the translational constraints remain in place. 

Masses of viscerae, below the diaphram, are included as 8 additional rigid
bodies, connected to the corresponding vertebral nodes with linear viscoelastic
elements, as well.

The pelvic region is modeled taking into account the compliance of the buttocks
tissue, by introducing two viscoelastic elements connecting the Sacrum with a
node representing the mean interface point between the buttocks and the resting
surface.

Geometry and inertial parameters are adapted to represent a generic subject possessing
the desired anthropometric characteristics of age, gender, stature and weight,
as it will be described in detail in the next section.

